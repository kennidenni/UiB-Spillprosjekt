% !TEX encoding = UTF-8 Unicode
%%%%%%%%%%%%%%%%%%%%%%%%%%%%%%%%%%%%%%%%%%%%%%%%%%%%%%%%%%%%%%%%%%%%%
% LaTeX Template: Project Titlepage Modified (v 0.1) by rcx
%
% Original Source: http://www.howtotex.com
% Date: February 2014
% 
% This is a title page template which be used for articles & reports.
% 
% This is the modified version of the original Latex template from
% aforementioned website.
% 
%%%%%%%%%%%%%%%%%%%%%%%%%%%%%%%%%%%%%%%%%%%%%%%%%%%%%%%%%%%%%%%%%%%%%%

\documentclass[12pt]{report}
\usepackage[a4paper]{geometry}
\usepackage{fancyhdr}
\usepackage{lastpage}
\usepackage{hhline}

%-------------------------------------------------------------------------------
% HEADER & FOOTER
%-------------------------------------------------------------------------------
\pagestyle{fancy}
\fancyhf{}
\setlength\headheight{15pt}
\fancyhead[L]{Team D}
\fancyhead[R]{Universitetet i Bergen}
\fancyfoot[R]{Page \thepage\ of \pageref{LastPage}}
%-------------------------------------------------------------------------------
% TITLE PAGE
%-------------------------------------------------------------------------------

\begin{document}

%-------------------------------------------------------------------------------
% BODY
%-------------------------------------------------------------------------------

\section*{Organisering av Oblig 4}

\subsection*{Utviklingsmetode}

Vi har valgt den smidige utviklingsmetoden SCRUM, og gjennomføre det som normalt. 
Vi kommer til å ha hyppige Sprints, hvor vi forteller om fremgang og videre plan, 
for å holde styr på hvordan folk ligger an, og for å eventuelt kunne reallokere arbeidskraft.

\subsection*{Oppdeling i grupper}

Vi ønsker å dele opp teamet i grupper, hvor en gruppe utvikler backend, en gruppe lager GUI, osv.

\subsection*{Evaluering av fremgang}

Vi kommer til å evaluere fremgang og forutse gjenstående arbeid 
ved hjelp av en burn-down chart som vil bli oppdatert ved hver Sprint.

\subsection*{Risikoanalyse}

Hvis vi skulle møte på problemer som f.eks. at folk slutter, eller ikke møter opp, 
vil vi først se på muligheten om å reallokere arbeidskraft, 
og hvis det ikke går vil vi prøve å fjerne trivielle funksjoner fra spillene. 
Hvis det ikke går vil vi felles avgjøre et spill som ikke blir implementert.

\subsection*{Felles funksjonalitet}

Felles egenskaper for alle spillene inkluderer GUI, 
innhenting av åpen data (vær) og elementer i spillet.
Disse elementene inkluderer highscore, items og andre objekter i spillene.

\subsection*{Vanskelighetsgradvurdering}

For å definere de viktigste/vanskeligste delene av implementasjonen
kommer vi til ha en planning poker i starten av Oblig 4.
Dette vil være lettere å definere i plenum.


\end{document}