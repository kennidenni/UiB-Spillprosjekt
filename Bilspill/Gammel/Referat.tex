\documentclass[paper=a4]{article}
\usepackage{ucs}
\usepackage[utf8x]{inputenc}
\usepackage[T1]{fontenc}
\PreloadUnicodePage{0}
\usepackage{xspace}
\usepackage{array}
\usepackage[hmargin=3.5cm,vmargin=2.7cm]{geometry} 



\title{Veldig gode spillideer!}
\begin{document}
\maketitle

Notater 14. Februar: \\
- Forskjellige ideer: \\
- 2D grave/ bygge -spill \\
- Fritidsspill mobil(type flappy bird, byggespill, temple run) \\
- 3D shooter \\

Beste ideen: \\
- 2D top-down bilspill, enten mobil eller pc (subway surfer style) \\
- Styrer med å holde til siden, bremse med å holde tilbake \\
- Hindringer i veien (BI folk, ting, bybane, hull) \\
- Kjøreegenskaper ut fra været \\
- Natt/dag \\
- Penger til oppgraderinger \\
- Plukke opp ting på veien (penger, bensin, boost, annet) \\
- Stor målgruppe, velge når vi vet mer om hvordan det skal være \\
Til neste gang: \\
- Fant ut hva vi skulle gjøre til neste gang \\


Notater 15. Februar:  \\
Avklarte diverse ting angående spillet. Delte ut arbeidsoppgaver fra oppgaveteksten (del 2.1) på personene i gruppene: \\
Joakim - pkt 1,2,3 og 6 \\
Peder pkt 4 og 8 \\ 
Natalie pkt 7 og 8 \\
Kenneth pkt 5 og 6 \\

Møtes igjen neste mandag \\

Notater 20. Februar: \\
Folk begynner å bli ferdig med ulike deler av oppgaven, de legges sammen og pushes til repo. \\
Mangler fortsatt noen punkter, men møtes igjen tirsdag i gruppen og kanskje onsdag for å gjøre ting ferdig. \\

Notater 21. Februar: \\
Kenneth var på lab, lab-leder sa det såg veldig bra ut! \\

Notater 22. Februar: \\
Samles i lesesalen igjen. La til de siste bildene i innleveringen, pushed det opp. \\
Så gjennom til vi var fornøyde med det vi hadde laget.  \\
Spurte gruppeleder om functional, non-functional, sa at det ikke trengs, da det ikke stod noe om det i oppgaven.

\end{document}
