\documentclass[a4paper]{article}
\usepackage[utf8]{inputenc}

\begin{document}

\title{Referat gruppearbeid INF112}
\date{15.2.17}

\author{Referent: Mats Ryland}
\maketitle


\section{Oppmøte:}
Oppmøte: Malin, Mats, Hans Ivar, og Eirik

\section{Spillidé}
\subsection{Horrospill}
\section{Diskusjon}

\begin{itemize}
    \item{Point and click}
    \item{Horror}
    \item Point and click
    \item Horror
    \item Vi befinner oss i et hus
    \item Psykologisk skrekk/kveppe-horror
    \item Historiedrevet
\end{itemize}

\section{Målgruppe:}
16/17+ \\
Avhengig av hvor skummelt det blir.\\
\section{Hvor omfattende er spillet?}
Antall rom/brett\\
Tidspress\\
Historie? Hvor befinner vi oss?\\

\section{Spilltype:}
Spiller mot maskin\\

\section{Spillmotor?}

\subsection{2d}
\begin{itemize}
    \item Unity
\end{itemize}

\subsection{3d}
\begin{itemize}
    \item Unreal
    \item Unity
    \item JMonkeyEngine
\end{itemize}

\section{Spillmekanikker:}
\begin{itemize}
    \item Point and click navigasjon
    \item Tidspress som stressfaktor
    \item Lyskilde som går tom for batteri
    \item Groundhogday?? -->Skrekkspill
    \item Mister noen/alle gjenstander hver runde.
    \item Introduser nye mekanikker hver gang runden starter, etter løste problem.
\end{itemize}

\section{Historie}
\end{document}
