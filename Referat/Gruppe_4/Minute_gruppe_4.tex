\documentclass[norsk,a4paper]{meetingmins}
\setcommittee{Minutes}
\usepackage{babel}
\usepackage[utf8x]{inputenc}
\usepackage[T1]{fontenc}
\PreloadUnicodePage{0}

\setdate{March, 15}

\setpresent{
  Philip, Kenneth, Emilia, Eirik
}

\absent{none\textit{}}


\begin{document}
\maketitle

\paragraph{Hva vi diskuterte og hva vi fikk gjort dette møtet:}

\begin{items}
\item
Bestemte for hvilket spill hver av oss skulle jobbe med
\item
Tok i utgangspunktet brukerfortellingene til hvert spill for å lage brukerkravspesifikasjon
\item
Kom med flere ideer til brukerfortellingene 
\item Analyserte brukerfortellingene og delte den opp i flere deler og delte dem opp i følgende:
	\begin {items}
		\item Aktør 
		\item Ønske
		\item Grunn
    \end{items}
\item
Brukte i utgangspunktet MoSCoW metoden, men med kun MustHave og ShouldHave og andre krav som funksjonell og ikke-funksjonell
\item
Viste fram hva vi hadde gjort og diskuterte kravene om de var enten MustHave/ShouldHave og Funksjonell/Ikke-funksjonell
\end{items}

\subsection{Tasks for/to next meeting}
\begin{items}
\item
Gå gjennom filene
\item
Presentere til neste møte
\end{items}

\vspace{1em}
\nextmeeting{Tuesday, 16 march, at 12:00 pm}

\end{document}

