\documentclass[11pt]{article}
\usepackage[utf8x]{inputenc}
\usepackage[T1]{fontenc}

\begin{document}

\section*{Start av Oblig4}
Disse metodene skulle vi bruke i Oblig 4:
\begin{itemize}
\item Scrum
\item Parprogrammering
\item Burndown chart
\item Planning poker
\item Skrive tester
\end{itemize}

\vspace{1em}

\noindent Etter Oblig 3 hadde vi bestemt oss for å bruke Scrum som
utviklingsmetode der vi skulle ha fast møte 1 gang i uken.
Vi skulle sitte to eller tre i lag som jobbet med det samme slik at vi tok i bruk parprogrammering.
Alle skulle også skrive tester til sine egne metoder. 
Vi prøvde å bruke Burndown chart, men vi fant ingen god måte å bruke det på. 
Burndown chart ble da droppet. Vi ble også enige om å bruke Planning Poker slik
at vi kunne fordele arbeidet ganske likt mellom personene i gruppen.
Vi skulle fortsette med Slack for å snakke i lag, og vi skulle bruke Trello for å ha Scrum boards. 
Vi skulle bruke Pipeline i Gitlab, men Håvar holdt fortsatt på med å sette dette
opp.

\section*{Sprint 1}
Vi delte oss inn i par eller tre og tre. Noen skulle jobbe med GUI, andre med generelle klasser og andre med tegning.

\section*{Sprint 2}
Vi var ikke veldig flinke til å bruke Trello på riktig måte, men det gjorde ikke
så mye ettersom Håvar fant ut at det var bedre å bruke Issues på Gitlab. Det var
lettere og vi kunne få Scrum-lignende boards.
Vi fant også ut at Planning Poker tok litt lang tid i forhold til den tiden vi
hadde på denne obligen. Vi endte derfor opp med å bare diskutere hvor vanskelig
det var og så fordele oppgavene på de som hadde jobbet med noe lignende i Sprint 1.

\section*{Sprint 3}
Det var nå organiseringen kom skikkelig på plass og alle hadde blitt vant med hvordan arbeidsflyten vår var. Nå fordelte vi ikke lenger oppgavene på samme måte som tidligere fordi alle oppgavene som skulle bli gjort lå på git. 
Dette betydde at du selv valgte et issue som du ville jobbe med, og skrev at du
holdt på med det.

\section*{Sprint 4}
Organiseringen fungerer på samme måte som i Sprint 3. 

\section*{Hvordan vi er organisert}

Vi valgte å dele oss inn i to grupper: Programmerings gruppe og Art gruppe.
Vi brukte en forenklet git flow metode som er bedre forklart i CONTRIBUTION
GUIDE på git-Prosjektet, hvor alt baserer seg på å løse issues.

\vspace{1em}

\noindent Vi fastsatte 5 roller som hjalp med å gi bedre kommunikasjonsflyt og
oversikt:
\begin{itemize}
\item Kommunikasjonansvarlig
\item Git-ansvarlig
\item Juridisk ansvarlig
\item Kode ansvarlig
\item CoachGunnar
\end{itemize}

\section*{Ting som kunne bli løst bedre}
\begin{itemize}
\item Dårlig prioritering av issues
\item Definering av arbeidsoppgaver
\item Dårlig label setting på issues
\end{itemize}

\vspace{1em}

\end{document}