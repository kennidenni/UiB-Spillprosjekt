\documentclass[11pt]{meetingmins}
\usepackage[utf8x]{inputenc}
\usepackage[T1]{fontenc}

\begin{document}
\subsection{Start av oblig4}
Vi skulle bruke disse metodene i oblig4.
\begin{items}
\item Scrum
\item Parprogrammering
\item Burndown chart
\item Planning poker
\item Skrive tester
\end{items}

Etter oblig3 hadde vi bestemt oss for å bruke Scrum som utviklingsmetode der vi skulle ha fast møte 1 gang i uken. Vi skulle sitte to eller tre i lag som jobbet med det samme slik at vi tok i bruk parprogrammering. Alle skulle også skrive tester til sine egne metoder. Vi prøvde å bruke Burn-down chart, men vi fant ingen god måte å bruke det på. Dette betydde at vi droppet dette. Vi ble også enige om å bruke planning poker slik at vi kunne fordele arbeidet ganske likt mellom personene i gruppen. Vi skulle fortsette med Slack for å snakke i lag, og vi skulle bruke Trello for å ha Scrum boards. Vi skulle bruke Pipeline i Gitlab, men Håvar holdt fortsatt på med å sette opp 

\subsection{Sprint 1}
Vi delte oss inn i par eller tre og tre. Noen skulle jobbe med GUI, andre med generelle klasser og andre med tegning.

\subsection{Sprint 2}
Vi var ikke veldig flinke til å bruke trello på rett måte, men det gjorde ikke så veldig mye fordi Håvar fant ut at det var veldig smart å bruke Issues på Gitlab. Det var mye greiere og vi kunne få Scrum-lignende boards. Vi fant også ut at planning poker tok litt lang tid i forhold til den tiden vi hadde på denne obligen. Vi endte derfor opp med å bare diskutere hvor vanskelig det var og så fordele oppgavene ut på de som hadde jobbet med noe lignende i sprint1.

\subsection{Sprint 3}
Det var først nå at organiseringen kom skikkelig på plass og alle hadde blitt vandt med hvordan arbeidsflyten vår var. Nå fordelte vi ikke lenger oppgavene på samme måte som tidligere fordi alle oppgavene som skulle bli gjort låg på git. Dette betydde at du selv valgte et Issue som du ville jobbe med, og skrev at du holdt på med det. 

\subsection{Sprint 4}
Organiseringen fungerer på samme måte som i Sprint 3. 

\vspace{1em}
\end{document}
