% !TEX encoding = UTF-8 Unicode
%%
%% This is file `./samples/minutes.tex',
%% generated with the docstrip utility.
%%
%% The original source files were:
%%
%% meetingmins.dtx  (with options: `minutes')
%% ----------------------------------------------------------------------
%% 
%% meetingmins - A LaTeX class for formatting minutes of meetings
%% 
%% Copyright (C) 2011-2013 by Brian D. Beitzel <brian@beitzel.com>
%% 
%% This work may be distributed and/or modified under the
%% conditions of the LaTeX Project Public License (LPPL), either
%% version 1.3c of this license or (at your option) any later
%% version.  The latest version of this license is in the file:
%% 
%% http://www.latex-project.org/lppl.txt
%% 
%% Users may freely modify these files without permission, as long as the
%% copyright line and this statement are maintained intact.
%% 
%% ----------------------------------------------------------------------
%% 
\documentclass[11pt]{meetingmins}
\usepackage[utf8]{inputenc}
\usepackage[T1]{fontenc}
%% Optionally, the following text could be set in the file
%% department.min in this folder, then add the option 'department'
%% in the \documentclass line of this .tex file:
%%\setcommittee{Department of Instruction}
%%
\setcommittee{Minutes}
\setdate{April, 3}
\setpresent{
  Phillip, Kenneth, H{\aa}var, Peter, Malin, Eirik, Markus, Amund, Ole Magnus, Anna, Torbj{\o}rn, Emilia
}
\setabsent{
  Sturle, Ola, Bj{\o}rnar
}
\begin{document}
\maketitle
\subsection{Saksliste}

\begin{itemize}
\item Identifisere oppgaver
\item Planning poker for {\aa} finne vanskelighetsgrad av oppgaver
\item Sette opp SonarQube server
\item Sette opp prosjektet p{\aa} sin egen datamaskin slik at alle er klar
\end{itemize}

\subsection{Referat}

\begin{itemize}
\item De som ikke har lest oppgaveteksten, leser den.
\item Phillip skrev sammendrag av hva vi m{\aa} gj{\o}re i oblig 4
\item H{\aa}var har satt opp en SonarQube server som er fungerende på sin egen maskin
\item Sm{\aa} endringer i gradle filer for {\aa} f{\aa} den kompitabel med IntelliJ
\item G{\aa}r gjennom sammendraget til oblig 4
\item For neste leveranse skal vi lage en {\textquotedbl}pre-alpha{\textquotedbl} versjon for {\aa} kunne ha et spill
som er fungerende og som vi kan bygge videre p{\aa}.
\end{itemize}

\begin{itemize}
\item Milep{\ae}l

\begin{itemize}
\item Designmodell
\item API
\item Diagrammer
\end{itemize}
\item Hva vi m{\aa} ha med for {\aa} f{\aa} fungerende spill

\begin{itemize}
\item Main metoder til hvert spill
\item Hvis vi ser p{\aa} Designmodell, s{\aa} er det alt fra Game til GameScreen og videre
\item Game, GameScreen, Layer, GameObject, Item, Actor
\end{itemize}
\item Resultat av session med planning poker: \\
 Actor - 2 \\
 Game - 13 \\
 GameObject - 5 \\
 GameScreen - 13 \\
 Main metoder for {\aa} startet hvert spill - 2 \\
 Item - 1 \\
 Layer - 5 \\
\newline
Disse klassene er felles for alle spill, bortsett fra Main metodene, slik at vi har en liten versjon vi kan implementere videre p{\aa} n{\aa}r vi skal jobbe med de andre spillene.

\item Grafikk

\begin{itemize}
\item Skal lage alle illustrasjoner og lyder selv
\item Verkt{\o}y for illustrasjon: Piskel
\item Spooks

\begin{itemize}
\item Lager veldig simpel grafikk i starten, og s{\aa} utvide det
\item Fylle rommet bare med farger for forskjellige objekter, s{\aa} bygge p{\aa} med bedre illustrasjoner etterhvert
\item Begynne med noen f{\aa} rom i starten
\item Rommet blir i 2D
\item Det skal ikke v{\ae}re mulig {\aa} dra objektet rundt i rommet, n{\aa}r man trykker p{\aa} den vil vi f{\aa} den i
Inventory
\item Inventory skal v{\ae}re en bare nederst p{\aa} skjermen (Slik som minecraft)
\end{itemize}
\end{itemize}
\item Vi vil sp{\o}rre IT-avdelingen p{\aa} UiB om vi kan ha SonarQube p{\aa} UiB sin server, ellers bruker vi amazon
sin gratisperiode
\end{itemize}

\subsection{Saksliste til neste m{\o}te}

\begin{itemize}
\item Dele oss i mindre lag
\item Diskutere evt. flere oppgaver
\end{itemize}

\nextmeeting{Onsdag, 5. April, kl 14:15}
\end{document}
%% 
%% Copyright (C) 2011-2013 by Brian D. Beitzel <brian@beitzel.com>
%% 
%% This work may be distributed and/or modified under the
%% conditions of the LaTeX Project Public License (LPPL), either
%% version 1.3c of this license or (at your option) any later
%% version.  The latest version of this license is in the file:
%% 
%% http://www.latex-project.org/lppl.txt
%% 
%% Users may freely modify these files without permission, as long as the
%% copyright line and this statement are maintained intact.
%% 
%% This work is "maintained" (as per LPPL maintenance status) by
%% Brian D. Beitzel.
%% 
%% This work consists of the file  meetingmins.dtx
%% and the derived files           meetingmins.cls,
%%                                 sampleminutes.tex,
%%                                 department.min,
%%                                 README.txt, and
%%                                 meetingmins.pdf.
%% 
%%
%% End of file `./samples/minutes.tex'.