\documentclass[norsk,a4paper]{article}
% Setter orddelingsregler for norsk (se opsjon "norsk" over)
\usepackage{babel}
% Setter UTF-8 tegnsett for latex (et passende utvalg fra Unicode for norsk)
\usepackage{ucs}
\usepackage[utf8x]{inputenc}
\usepackage[T1]{fontenc}
\PreloadUnicodePage{0}
% Håndterer mellomrom på en smidig (og automatisk) måte.
\usepackage{xspace}
% Gjør lenker (også internt i dokumentet) klikkbare
\usepackage{hyperref}
% God formatering av URL
\usepackage{url}
% Gjør farger tilgjengelig, feks fargede bokstaver og strektegninger
\usepackage{color}
% Kommandoer for å integrere bilder i teksten
\usepackage{graphicx}
% Justering av margene (mindre enn standard oppsett)
\usepackage[hmargin=3.5cm,vmargin=2.7cm]{geometry} 

% Tittel, forfatter(e)
\title{Brukerkravspesifikasjon for TD-Uteliv}
\begin{document}
% Legger tittelen først i dokumentet
\maketitle
% Lager en innholdsfortegnelse - kommentert ut
% \tableofcontents
% Starten på selve teskten

\begin{center}
\begin{tabular}{ | p{0,75cm} | p{6cm} | p{2cm} | p{3cm} |}
    \hline
    Nr. & Krav & MåHa/BørHa & Funksjonell/Ikke-funksjonell \\ \hline
    1. & Som en \textbf{bruker} skal jeg kunne starte med en viss sum med penger og tjene opp penger per runde i løpet av spillet slik at jeg kan kjøper/oppgradere tårn & MåHa & Funksjonell \\ \hline
    
    2. & Som en \textbf{bruker} skal jeg kunne kjøpe tårn, samt plassere tårn på banen/kartet for å angripe fiender og beskyttet hovedtårnet mitt slik at jeg ikke mister liv 
 & MåHa &
    Funksjonell \\ \hline
    
    3. & Som en \textbf{bruker} skal jeg begynne med 10 liv for hvert spill jeg starter slik at jeg kan tape med en gang
 & MåHa & Funksjonell \\ \hline
    
   	4. & Som en \textbf{bruker} skal jeg kunne miste liv når fiender kommer til målet sitt slik at jeg er et steg nærmere å tape & MåHa & Funksjonell \\ \hline
    
    5. & Som en \textbf{bruker} skal jeg kunne tape når jeg har mistet alle mine liv slik at spillet blir avsluttet  & MåHa & Funksjonell \\ \hline
    
    6. & Som en \textbf{bruker} skal jeg ikke kunne plassere tårn på løypen/stien slik at jeg hindre veien for fiender & MåHa & Funksjonell \\ \hline
    
    7. & Som en \textbf{bruker} skal jeg ikke kunne kjøpe et tårn dersom jeg ikke har nok penger slik at jeg ikke kan kjøpe ubegrenset med tårn og at spillet ikke blir for lett 
 & MåHa & Funksjonell \\ \hline
    
    8. & Som en \textbf{bruker} bør jeg kunne oppgradere mine tårn slik at tårnene gjør mer skade, samt øke rekkevidden på tårnene slik at jeg kan enklere bekjempe fiender & MåHa &
    Funksjonell \\ \hline
     
    9. & Som en \textbf{bruker} ønsker jeg å kunne få en pengebelønning for å fullføre en runde slik at kan bruke penger på å kjøpe tårn eller oppgradere & BørHa & Funksjonell \\ \hline
     
    10. & Som en \textbf{bruker} vil jeg at systemet skal øke vanskelighetsgraden for hver runde slik at det blir vanskeligere for å komme seg videre i spillet & BørHa & Funksjonell \\ \hline
   
    11. & Som en \textbf{bruker} vil jeg at det skal komme nye og sterke fiender jo lenger jeg kommer ut i spillet slik at det vil kreve mer for å drepe fiendene & BørHa & Funksjonell
    \\ \hline
    
    
\end{tabular}

\begin{tabular}{ | p{0,75cm} | p{6cm} | p{2cm} | p{3cm} |}
    \hline   
     
    12. & Som en \textbf{bruker} ønsker jeg at belønningen for å drepe fiender skal øke i stil med rundens vanskelighetsgrad slik at jeg kan få mer penger for å kunne kjøpe flere tårn eller oppgraderinger & BørHa & Funksjonell \\
    \hline
    
    13 & Som en \textbf{bruker} vil jeg kunne lagre framgangen min slik at jeg kan spille videre fra der jeg avsluttet fra & BørHa & Funksjonell \\ \hline
      
    14 & Som en \textbf{bruker} skal jeg ikke kunne lagre framgangen min midt i en runde slik at jeg ikke skal kunne begynne midt i en runde når jeg starte opp framgangen min igjen  & BørHa & Funksjonell \\ \hline
   
    15 & Som en \textbf{bruker} vil jeg at spillet skal bli påvirket av ulike værforhold i samsvar med data fra yr.no slik at det skaper variasjoner i spillet  & MåHa & Funksjonell \\ \hline
    
    16 & Som en \textbf{bruker} vil jeg at spillet skal samsvare med dager i yr.no sin langtidsvarsel hvor hver enkelte dag i uken skal representere en runde i spillet slik at det blir flere værforhold i spillet & BørHa & Funksjonell \\ \hline
    
    17 & Som en \textbf{bruker} vil jeg at stien/løypen skal bli glattere når det regner slik at fiender vil kunne gå raskere slik at det blir vanskeligere & BørHa & Funksjonell \\ \hline
    
    18 & Som en \textbf{bruker} vil jeg at fiendene skal gå tregere når det snør, men det skal også være glattere noen steder på stien/løypen slik at det er ulike variasjoner etter værforhold & BørHa & Funksjonell \\ \hline
    
    19 & Som en \textbf{bruker} skal jeg kunne fullføre spillet etter å ha kommet meg gjennom et viss antall med runder slik at jeg kan vinne spillet & MåHa & Funksjonell \\ \hline
    
    20 & Som en \textbf{bruker} vil jeg kunne velge ulike baner slik at jeg kan spille på andre baner & BørHa & Funksjonell \\ \hline
    
    21 & Som en \textbf{bruker} skal jeg kunne få starte spillet på nytt eller avsluttet spillet når jeg taper slik at jeg kan få enten begynne på nytt eller gjøre noe annet  & BørHa & Funksjonell \\ \hline
       
   22 & Som en \textbf{bruker} skal jeg kunne pause/stoppet spillet når som helst i spillet & BørHa & Funksjonell \\ \hline
   
   23 & Som en \textbf{bruker} ønsker jeg å kunne få poeng for hver fiende jeg dreper og poeng for hver runde jeg fullføre slik at jeg kan få en poengscore  & BørHa & Funksjonell 
   \\ \hline
   
   24 & Som en \textbf{bruker} vil jeg at poengscoren skal variere etter hvor fort jeg klare å fullføre en runde slik at jeg vil få en høyere poengscore dersom jeg fullføre en runde raskt & BørHa & Funksjonell \\ \hline 
    
\end{tabular}
\end{center} 
\end{document}