%%%%%%%%%%%%%%%%%%%%%%%%%%%%%%%%%%%%%%%%%%%%%%%%%%%%%%%%%%%%%%%%%%%%%
% LaTeX Template: Project Titlepage Modified (v 0.1) by rcx
%
% Original Source: http://www.howtotex.com
% Date: February 2014
% 
% This is a title page template which be used for articles & reports.
% 
% This is the modified version of the original Latex template from
% aforementioned website.
% 
%%%%%%%%%%%%%%%%%%%%%%%%%%%%%%%%%%%%%%%%%%%%%%%%%%%%%%%%%%%%%%%%%%%%%%
\documentclass[12pt]{report}
\usepackage[a4paper]{geometry}
\usepackage[myheadings]{fullpage}
\usepackage{fancyhdr}
\usepackage{lastpage}
\usepackage{graphicx, wrapfig, subcaption, setspace, booktabs}
\usepackage[english]{babel}
\usepackage{color}
\usepackage{hyperref}
\usepackage{array}
\usepackage{supertabular}
\usepackage{hhline}
\usepackage{enumitem}
\usepackage[T1]{fontenc}
\usepackage[utf8]{inputenc}
\usepackage{graphicx}
\newcommand{\HRule}[1]{\rule{\linewidth}{#1}}
\renewcommand{\theenumii}{\arabic{enumii}.}
\addto\captionsenglish{
  \renewcommand{\contentsname}
    {Innhold}
}
\onehalfspacing
\setcounter{tocdepth}{5}
\setcounter{secnumdepth}{5}
%-------------------------------------------------------------------------------
% HEADER & FOOTER
%-------------------------------------------------------------------------------
\pagestyle{fancy}
\fancyhf{}
\setlength\headheight{15pt}
\fancyhead[L]{Team D} 
\fancyhead[R]{Universitetet i Bergen}
\fancyfoot[R]{Page \thepage\ of \pageref{LastPage}}
%-------------------------------------------------------------------------------
% TITLE PAGE
%-------------------------------------------------------------------------------
\begin{document}
\title{ \normalsize \textsc{}
		\\ [2.0cm]
		\HRule{0.5pt} \\
		\LARGE \textbf{\uppercase{TD Uteliv }}
		\HRule{2pt} \\ [0.5cm]
		\normalsize \today \vspace*{5\baselineskip}}
\date{}
\author{
		Team D  \\ 
		Universitetet i Bergen \\
		Informatikk }
\maketitle
\tableofcontents
\newpage
%-------------------------------------------------------------------------------
% BODY
%-------------------------------------------------------------------------------
\section*{Bruksm{\o}nstertekst:}
\addcontentsline{toc}{section}{Bruksm{\o}nstertekst:}
\textbf{Tittel}: Få motstandere til å gå hjem.
\bigskip \\
\textbf{Akt{\o}rer}: Spillere, system
\bigskip \\
\textbf{Prim{\ae}rakt{\o}r}: Spiller
\bigskip \\
\textbf{Tid}: 45 sekunder før fiender genereres, runden varer frem til alle fiendene er av brettet eller en fiende overlever siste utested. 
\bigskip \\
\textbf{M{\aa}l}: Sjenke alle motstandere slik at de blir fulle og går hjem. 
\bigskip \\
\textbf{Pre-conditions:} Spillet er startet p{\aa} en datamaskin
\subsubsection*{Hovedflyt:}
\begin{enumerate}
\item Systemet genererer banen. 
\item Aktøren får 45 sekunder til å sette ut og oppgradere utesteder. 
\item Systemet genererer fiender som går langs banen. 
\item Systemet beregner skade utgjort av de forskjellige utestedene. 
\item Systemet beregner fiendens helse ved hvert utested.  
\item Dersom en fiende får ferre enn ett helsepoeng så går den tilbake.  
\item En fiende overlever siste utested, og spiller taper. 
\end{enumerate}
\subsubsection*{Alternativ handlinger:}
\begin{enumerate}[label=\Alph*]
\item 
\bigskip
\begin{enumerate}
\item @7 Ingen motstandere overlever siste utested.
\item Gjennoppta @2. Men @3 genereres det fiender med flere helsepoeng.  
\end{enumerate}
\end{enumerate}
\section*{Bruksm{\o}nsterdiagram:}
\addcontentsline{toc}{section}{Bruksm{\o}nsterdiagram:}
\vspace{1cm}
\includegraphics[width=500px]{TDutelivDiagram.jpg}
\end{document}