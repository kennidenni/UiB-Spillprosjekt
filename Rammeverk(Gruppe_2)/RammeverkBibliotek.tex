% !TEX encoding = UTF-8 Unicode
%%%%%%%%%%%%%%%%%%%%%%%%%%%%%%%%%%%%%%%%%%%%%%%%%%%%%%%%%%%%%%%%%%%%%
% LaTeX Template: Project Titlepage Modified (v 0.1) by rcx
%
% Original Source: http://www.howtotex.com
% Date: February 2014
% 
% This is a title page template which be used for articles & reports.
% 
% This is the modified version of the original Latex template from
% aforementioned website.
% 
%%%%%%%%%%%%%%%%%%%%%%%%%%%%%%%%%%%%%%%%%%%%%%%%%%%%%%%%%%%%%%%%%%%%%%

\documentclass[12pt]{report}
\usepackage[a4paper]{geometry}
\usepackage[myheadings]{fullpage}
\usepackage{fancyhdr}
\usepackage{lastpage}
\usepackage{graphicx, wrapfig, subcaption, setspace, booktabs}
\usepackage[english]{babel}
\usepackage{color}
\usepackage{hyperref}
\usepackage{array}
\usepackage{supertabular}
\usepackage{hhline}

\newcommand{\HRule}[1]{\rule{\linewidth}{#1}}

\addto\captionsenglish{
  \renewcommand{\contentsname}
    {Innhold}
}

\onehalfspacing
\setcounter{tocdepth}{5}
\setcounter{secnumdepth}{5}

%-------------------------------------------------------------------------------
% HEADER & FOOTER
%-------------------------------------------------------------------------------
\pagestyle{fancy}
\fancyhf{}
\setlength\headheight{15pt}
\fancyhead[L]{Team D}
\fancyhead[R]{Universitetet i Bergen}
\fancyfoot[R]{Page \thepage\ of \pageref{LastPage}}
%-------------------------------------------------------------------------------
% TITLE PAGE
%-------------------------------------------------------------------------------

\begin{document}

\title{ \normalsize \textsc{}
		\\ [2.0cm]
		\HRule{0.5pt} \\
		\LARGE \textbf{\uppercase{Forslag til rammeverk,
		biblioteker og konvensjoner}}
		\HRule{2pt} \\ [0.5cm]
		\normalsize \today \vspace*{5\baselineskip}}

\date{}

\author{
		Team D \\ 
		Universitetet i Bergen \\
		Informatikk }

\maketitle
\tableofcontents
\newpage

%-------------------------------------------------------------------------------
% BODY
%-------------------------------------------------------------------------------

\section*{Bygningsprosessen}
\addcontentsline{toc}{section}{Bygningsprosessen}

Bygningsprosessen definerer vi ved hjelp av Gradle. F{\ae}rrest mulige steg fra {\aa} laste ned prosjektet til det er
ferdig bygget skal oppn{\aa}s. Hvis vi f{\aa}r dette til p{\aa} en god m{\aa}te blir det mye lettere {\aa} sette opp og
jobbe med prosjektene og vanskeligere {\aa} {\o}delegge prosjektet med uhell slik at det ikke kan kompileres.\\

Bygningsprosessen Gradle utf{\o}rer vil g{\aa} omtrent slik:

\begin{itemize}
\item Kj{\o}re FindBugs
\item Kompilere Java-filene
\item Kj{\o}re JUnit
\item Sette sammen n{\o}dvendige filer til en kj{\o}rbar JAR-fil (eller andre format hvis spillene skal bli eksportert
til andre platformer)
\end{itemize}

\newpage
\section*{Konvensjoner}
\addcontentsline{toc}{section}{Konvensjoner}

For {\aa} redusere individuelle problemer som oppst{\aa}r p{\aa} grunn av forskjellige milj{\o} har vi bestemt oss for
{\aa} effektivisere prosessen av {\aa} laste ned prosjektet, importere det til en IDE og bygge og teste det. Dette har
vi gjort ved {\aa} automatisere s{\aa} mye av arbeidet som blir gjort mot prosjektet som mulig.\\

I tillegg har vi valgt noen standarder som m{\aa} bli fulgt for at koden og filene i prosjektet blir lettere {\aa} lese
og gjenkjenne.\\

\subsection*{Navn p{\aa} filer i prosjektet}
\addcontentsline{toc}{subsection}{Navn p{\aa} filer i prosjektet}

Navn p{\aa} filer i prosjektet, spesielt ressursfiler (lyder, bilder, osv{\dots}), skal v{\ae}re p{\aa} engelsk med
sm{\aa} bokstaver og understrek i stedet for mellom rom. Selv om selve filnavnene er p{\aa} engelsk kan innholdet
v{\ae}re p{\aa} norsk. Unntak av disse reglene kan forekomme hvis et verkt{\o}y forventer et spesifikt filnavn.\\

Noen eksempler:
\begin{itemize}
\item Et bilde av en basketball: \textit{basketball.png}
\item Et bilde av en trollfiende: \textit{troll\_enemy.png}
\item En lydfil av et smell: \textit{loud\_bang.ogg}
\end{itemize}

\subsection*{Googles kodekonvensjoner for Java}
\addcontentsline{toc}{subsection}{Googles kodekonvensjoner for Java}

Vi har valgt {\aa} bruke Google sine kodekonvensjoner for Java. Oracle sine var utdatert og manglet konvensjoner for
nyere funksjoner i spr{\aa}ket. I tillegg skal all kode og navn p{\aa} kildekodefiler v{\ae}re p{\aa} engelsk.\\

\href{https://google.github.io/styleguide/javaguide.html}{\textcolor{blue}{Googles kodekonvensjoner for Java}}


\newpage
\subsection*{Format og lesing av filer}
\addcontentsline{toc}{subsection}{Format og lesing av filer}

Formatet og lesing av eksterne filer burde ogs{\aa} v{\ae}re standardisert slik at det er likt gjennom hele prosjektet.
Dette gj{\o}r ogs{\aa} at vi kan gjenbruke klasser som leser filformat i stedet for {\aa} lage nye l{\o}sninger for
hver fil. Forslaget v{\aa}rt er f{\o}lgende: \\

\begin{center}
\begin{tabular}{ | m{5cm} | m{1cm} | m{4cm} | m{5cm} |}
\hline
Filtype & Format & Leser & Forklaring\\
\hline

Sm{\aa}-medium st{\o}rrelse bilder med gjennomsiktighet & PNG & LibGDX bildeklasser. & ~ \\
\hline

Store bilder uten gjennomsiktighet & JPG & LibGDX bildeklasser. &
Gjelder bilder som ikke trenger {\aa} bruke tapsfri komprimering. \\
\hline

Lydfiler (korte og lange) & OGG & LibGDX lydklasser. & ~ \\
\hline

Rene tekstfiler & TXT & Java sin IO-klasser. & ~ \\
\hline

Hierarkisk tekstfiler & JSON & GSON-biblioteket. &
Disse type filer kan brukes til {\aa} definere informasjon til elementer i spillene uten {\aa} rote til koden. For
eksempel navnet og egenskaper p{\aa} forskjellige karakterer i et spill. \\
\hline

XML (ogs{\aa} en hierarkisk tekstfil) & XML & W3 sine XML-klasser (f{\o}lger med JDK) &
Dette formatet skal bare brukes hvis n{\o}dvendig (for eksempel til YR sin v{\ae}rdata).\\
\hline
\end{tabular}
\end{center}

\newpage
\section*{Rammeverk og bibliotek}
\addcontentsline{toc}{section}{Rammeverk og bibliotek}

\subsection*{Produksjon}
\addcontentsline{toc}{subsection}{Produksjon}

\subsubsection*{LibGDX}
\addcontentsline{toc}{subsubsection}{LibGDX}

LibGDX er et omfattende rammeverk basert p{\aa} Lightweight Java Game Library (LWJGL) for {\aa} utvikle spill til web,
mobil og desktop. Det er laget for Java og best{\aa}r av et hovedbibliotek og flere mindre tilleggsbibliotek. Det har
eksistert lenge og er veldig mye brukt og har derfor mye dokumentasjon og andre ressurser tilgjengelig p{\aa} nett. \\

Kjernen til LibGDX tilbyr klasser som p{\aa} det lave niv{\aa}et tar seg av vindubehandling, brukerinndata, lyd,
grafikk, filsystem og nettverkslogikk. Alle disse modulene er bygget opp p{\aa} en slik m{\aa}te at omtrent alle
platformspesifikke operasjoner er skjulte og derfor kan utviklere enkelt bygge til forskjellige platform fra en
kodebase. \\

\href{https://libgdx.badlogicgames.com/}{\textcolor{blue}{Hjemmeside}} {\textperiodcentered}
\href{https://en.wikipedia.org/wiki/LibGDX}{\textcolor{blue}{Wikipedia}}

\subsubsection*{Box2D}
\addcontentsline{toc}{subsubsection}{Box2D}

Box2D er et fysikkbibliotek for C++ som simulerer stive kropper i 2D. Det kan simulere realistisk, kontinuerlig
kollisjon mellom mangekanter ved {\aa} p{\aa}virke kroppene med krefter, gravitasjon og friksjon. \\

Takket v{\ae}re arbeidet til LibGDX finnes det en Java-port som i bakgrunnen bruker de kompilerte C++ filene. P{\aa}
grunn av dette kj{\o}rer det bedre enn hvis det hadde v{\ae}rt kj{\o}rt i JVM. \\

\href{http://box2d.org/}{\textcolor{blue}{Hjemmeside}} {\textperiodcentered}
\href{https://en.wikipedia.org/wiki/Box2D}{\textcolor{blue}{Wikipedia}}

\subsubsection*{SLF4J}
\addcontentsline{toc}{subsubsection}{SLF4J}

Simple Logging Facade (SLF4J) er en abstraksjon av Java sitt innebygde loggingrammeverk (\textit{java.util.logging}) som
blant annet gir utviklere mer kontroll over hvilke alvorlighetsgrader som skal bli rapportert. \\

\href{https://www.slf4j.org/}{\textcolor{blue}{Hjemmeside}} {\textperiodcentered}
\href{https://en.wikipedia.org/wiki/SLF4J}{\textcolor{blue}{Wikipedia}}

\subsubsection*{GSON}
\addcontentsline{toc}{subsubsection}{GSON}

GSON er et Java-bibliotek av Google som tar seg av serialisering og deserialisering av Java-objekter og ren JSON. \\

\href{https://github.com/google/gson}{\textcolor{blue}{Hjemmeside}} {\textperiodcentered}
\href{https://en.wikipedia.org/wiki/Gson}{\textcolor{blue}{Wikipedia}}

\subsection*{Testing}
\addcontentsline{toc}{subsection}{Testing}

\subsubsection*{JUnit}
\addcontentsline{toc}{subsubsection}{JUnit}

JUnit er et rammeverk for testing av komponenter (enheter) i et program. Det er en sentral del av XP (Extreme
Programming). Testing best{\aa}r blant annet av {\aa}:

\begin{itemize}
\item Teste om objekter er like
\item Teste om vilk{\aa}r er sanne eller falske
\item Teste om objekter er null
\item Forvente at unntak oppst{\aa}r
\end{itemize}

\href{http://junit.org/junit4/}{\textcolor{blue}{Hjemmeside}} {\textperiodcentered}
\href{https://no.wikipedia.org/wiki/JUnit}{\textcolor{blue}{Wikipedia}}

\subsubsection*{Mockito}
\addcontentsline{toc}{subsubsection}{Mockito}

Mockito er et bibliotek for enhetstesting ved hjelp av "mock objects" i Java. Det brukes til {\aa} simulere et objekts oppf{\o}rsel i kontekst av programmet og kan derfor simulerere komplekse objekter i et kontrollert milj{\o} for {\aa} teste egenskaper som er tungvinte {\aa} teste ved kj{\o}retidsanalyse. \\

\href{http://site.mockito.org/}{\textcolor{blue}{Hjemmeside}} {\textperiodcentered}
\href{https://en.wikipedia.org/wiki/Mock_object}{\textcolor{blue}{Wikipedia}}

\subsubsection*{Sonarlint}
\addcontentsline{toc}{subsubsection}{Sonarlint}

En plugin i Eclipse og IntelliJ som kontrollerer kodestil i kildekoden fortl{\o}pende mens den skrives. Gir
tilbakemelding om reglene for god kodestil ikke blir overholdt. \\

\href{http://www.sonarlint.org/}{\textcolor{blue}{Hjemmeside}}

\subsubsection*{FindBugs}
\addcontentsline{toc}{subsubsection}{FindBugs}

Plugin i Eclipse og IntelliJ som gjennomf{\o}rer statisk kodeanalyse over valgt kodebase og kontrollerer dette opp mot
fastsatte regler med hensikt {\aa} oppdage ``code smell''.  \\

\href{http://findbugs.sourceforge.net/}{\textcolor{blue}{Hjemmeside}} {\textperiodcentered}
\href{https://en.wikipedia.org/wiki/FindBugs}{\textcolor{blue}{Wikipedia}}

\subsection*{Andre verkt{\o}y}
\addcontentsline{toc}{subsection}{Andre Verkt{\o}y}

\subsubsection*{IDE}
\addcontentsline{toc}{subsubsection}{IDE}

Eclipse eller IntelliJ m{\aa} brukes for {\aa} v{\ae}re sikker p{\aa} at alle verkt{\o}yene som blir brukt fungerer hos
alle. Ideen er at alle som produserer kode forholder seg til samme programmeringsmilj{\o} og overholder lik kodestil og
f{\aa}r hyppig tilbakemelding (short feedback loop) fra de forskjellige verkt{\o}yene.

\subsubsection*{IDE plugins}
\addcontentsline{toc}{subsubsection}{IDE plugins}

Til b{\aa}de Eclipse og IntelliJ finnes f{\o}lgende plugins:

\begin{itemize}
\item Sonarlint (statisk kodeanalyse)
\item Findbugs (statisk kodeanalyse). Kan ogs{\aa} inkluderes som en del av bygningsprosessen.
\item JUnit (enhetstesting). Kan ogs{\aa} inkluderes som en del av bygningsprosessen.
\item Gradle (definisjon av bygningsprosess). Inkludert i nyere versjoner av Eclipse.
\end{itemize}

\end{document}